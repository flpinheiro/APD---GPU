\documentclass[a4paper, 12pt]{article}

\usepackage[portuges]{babel}
\usepackage[utf8]{inputenc}
\usepackage{amsmath}
\usepackage{indentfirst}
\usepackage{graphicx}
\usepackage[colorinlistoftodos]{todonotes}
\usepackage{verbatim}
\usepackage{textcomp}
\usepackage{gensymb}
\definecolor{darkblue}{rgb}{0, 0, 0.5}
\usepackage[colorlinks=true,allcolors=black,citecolor=darkblue]{hyperref}%

\begin{document}

\begin{titlepage}
\begin{center}
\textbf{\LARGE Universidade de Brasília}\\[0.5cm] 
\textbf{\large Departamento de Ciência da Computação}\\[0.2cm]
\vspace{20pt}
\includegraphics{Logo_UnB.png}\\[1cm]

\par
\vspace{32pt}
\textbf{\LARGE Seminário de Arquitetura de processadores digitais - GPU }\\
\vspace{30pt}
\textbf {\Large Autor:}\\[0.2cm]
\Large {Pedro Oliveira		17/0163768}\\[0.1cm]

\end{center}

\par
\vfill
\begin{center}
{{\normalsize Brasília}\\
{\normalsize \today}}
\end{center}
\end{titlepage}

%Sumário
\newpage
\tableofcontents
\thispagestyle{empty}
%End Sumário

\newpage



\section{Resumo}
O seguinte artigo, busca apresentar um pouco sobre as GPUs e suas soluções.


\section{Introdução}
Utilizada pelas mais diversas plataformas, as GPU's ou unidades de processamento gráfico se tornaram uma peça fundamental nos sistemas computacionais \cite{owens2008gpu}. Originalmente utilizadas para otimizar o tempo de geração de gráficos 2D e 3D, imagens e vídeos de sistemas operacionais, as GPU's estão agora presentes em notebooks, smartphones, tablets, vídeo-games, e até mesmo em automóveis. E ainda, por ser um processador com grande poder de paralelismo, os fabricantes de GPU's também tem mantido o foco em outras áreas onde o processamento paralelo pode ser utilizado, como simulações por análise de elementos finitos e processamento de sinais. Com isso surgiu um novo termo, GP-GPU, que são GPU's de propósito geral. Existem GPU's que são produzidas sem saída de vídeo na placa, normalmente essas placas possuem vários chips de GPU's
destinados ao mercado de processamento de alto desempenho, como é o caso de algumas GPU's da arquitetura NVIDIA, como a NVIDIA Tesla. Atualmente o mercado de GPU's, está dominado por três grande fabricantes, Intel, NVIDIA e ATI, sendo que a Intel está mais voltado a produção de processadores gráficos de baixo custo, integrados à placas-mãe, enquanto que a NVIDIA e a ATI atuam nos demais segmentos \cite{machado}.

\section{Pipeline Tradicional}
Quando a GPU é 

\section{Arquitetura da GPU}

\subsection{Uma visão macro}

\subsection{A matriz dos processadores}


\bibliographystyle{unsrt}  
  \bibliography{references}

\end{document}